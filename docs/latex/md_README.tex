\href{https://app.travis-ci.com/Abhishek-Nalawade/Human_Obstacle_Detector}{\tt } \href{https://coveralls.io/github/Abhishek-Nalawade/Human_Obstacle_Detector?branch=feature_one}{\tt } \subsection*{\href{https://opensource.org/licenses/MIT}{\tt } }

\subsection*{Contributors}

1) \href{https://github.com/abhishek-nalawade}{\tt Abhishek Nalawade} Graduate Student of M.\+Eng Robotics at University of Maryland. 2) \href{https://github.com/iamjadhav}{\tt Aditya Jadhav} Graduate Student of M.\+Eng Robotics at University of Maryland.

\subsection*{Overview}

Obstacle detection has been widely studied in the field of Computer Vision. It is a crucial part of Self Driving Cars as well as Industrial Autonomous Systems. In the case of Self Driving Cars, a sub-\/task of Obstacle Detection can be Detection of Humans who become obvious and high risk obstacles for the vehicles. The real-\/time Human Obstacle Detection becomes increasingly difficult in noisy and chaotic environments and hence the algorithms deployed need to be highly efficient and robust to flimsy environmental variables. For example, color based approaches lack efficiency in unstable lighting environments. Deep Learning based approaches offer high efficiency but are extremely expensive to implement.

This project attempts to implement a Perception module for A\+C\+ME Robotics using high-\/quality software engineering practices such as the Agile Iterative Process, Object Oriented Programming, Pair Programming and Test Driven Development. We attempt to achieve robust human detection using Histogram of Oriented Gradients (H\+OG) feature descriptor combined with a Support Vector Machine Model. The algorithm uses H\+OG features extracted from the input data to detect Humans and surround them with rectangular bounding boxes, the centers of which act as the pixel coordinates of the detected humans. These coordinates combined with the pose matrix of the homogeneous transformation between the monocular camera and the 3D world will be used to track the detected humans. The pixel coordinates will be used for determining the distance of the humans with respect to the Robot’s Reference Frame. Support Vector Machine classifier will be used to train the extracted H\+OG features to train the training data.

The results of the Phase-\/1 are avilable in the Demo section. The technologies(libraries, tools, systems) used in order to build this project are listed in the next section.

\subsection*{Technology Used}

We will be following the Agile Iterative Software Development Process and the Test Driven Development Process. We will plan the work in Sprints and switch between the roles of driver and navigator throughout the process.


\begin{DoxyItemize}
\item Ubuntu 18.\+04 L\+TS
\item Modern C++ Programming Language
\item Open\+CV Library
\item Eigen Library
\item C\+Make Build System
\item Doxygen
\end{DoxyItemize}

\subsection*{Development}


\begin{DoxyItemize}
\item For this project we have followed Agile Iterative Development Process(\+A\+I\+P) and Test Driven Development Process(\+T\+D\+D).
\item The roles of Driver and Navigator were followed according to sprints of the project and switched between sprints.
\item Product Backlog and Work Log can be found \href{https://docs.google.com/spreadsheets/d/1bapR4zMCzfcwQHhxAm6KktWsMINTTPEt/edit#gid=2052063551}{\tt here}.
\item Sprint Planning and Overview document can be found \href{https://docs.google.com/document/d/1Xaz2rZ7OrmSh3bSE351XQz483VGetJkJdF37AjUF9Ro/edit}{\tt here}.
\end{DoxyItemize}

\subsection*{License}


\begin{DoxyCode}
MIT License

Copyright (c) 2021 Abhishek Nalawade, Aditya Jadhav

Permission is hereby granted, free of charge, to any person obtaining a copy
of this software and associated documentation files (the "Software"), to deal
in the Software without restriction, including without limitation the rights
to use, copy, modify, merge, publish, distribute, sublicense, and/or sell
copies of the Software, and to permit persons to whom the Software is
furnished to do so, subject to the following conditions:

The above copyright notice and this permission notice shall be included in all
copies or substantial portions of the Software.

THE SOFTWARE IS PROVIDED "AS IS", WITHOUT WARRANTY OF ANY KIND, EXPRESS OR
IMPLIED, INCLUDING BUT NOT LIMITED TO THE WARRANTIES OF MERCHANTABILITY,
FITNESS FOR A PARTICULAR PURPOSE AND NONINFRINGEMENT. IN NO EVENT SHALL THE
AUTHORS OR COPYRIGHT HOLDERS BE LIABLE FOR ANY CLAIM, DAMAGES OR OTHER
LIABILITY, WHETHER IN AN ACTION OF CONTRACT, TORT OR OTHERWISE, ARISING FROM,
OUT OF OR IN CONNECTION WITH THE SOFTWARE OR THE USE OR OTHER DEALINGS IN THE 
SOFTWARE.
\end{DoxyCode}


\subsection*{Demos and Overviews}

Please find the Phase-\/1 Overview here\+:
\begin{DoxyItemize}
\item \href{https://youtu.be/lwNjuT5e-FM}{\tt Phase-\/1}
\item \mbox{[}Phase-\/2\mbox{]}
\end{DoxyItemize}

\subsection*{Known Issues/\+Bugs}


\begin{DoxyItemize}
\item Human Detection is not highly efficient; Unnecessary bounding boxes are drawn for detections other than humans.
\item Humans in an abnormal pose are not being detected.
\end{DoxyItemize}

\subsection*{Dependencies}


\begin{DoxyItemize}
\item Install Open\+CV 3.\+4.\+4 and other dependencies using this link. Refer \href{https://learnopencv.com/install-opencv-3-4-4-on-ubuntu-18-04/}{\tt Open\+CV}
\end{DoxyItemize}

\subsection*{How to build}


\begin{DoxyCode}
git clone --recursive https://github.com/iamjadhav/Human\_Obstacle\_Detector
cd Human\_Obstacle\_Detector
mkdir build
cd build
cmake ..
make
\end{DoxyCode}


To Run tests 
\begin{DoxyCode}
./test/cpp-test
\end{DoxyCode}
 To Run the program/demo 
\begin{DoxyCode}
./app/shell-app
\end{DoxyCode}


\subsection*{Links}

A\+IP Process --$>$ \href{https://docs.google.com/spreadsheets/d/1bapR4zMCzfcwQHhxAm6KktWsMINTTPEt/edit#gid=2052063551}{\tt Link}

Sprint Planning Sheet--$>$ \href{https://docs.google.com/document/d/1Xaz2rZ7OrmSh3bSE351XQz483VGetJkJdF37AjUF9Ro/edit}{\tt Link}

Phase-\/1 Overview --$>$ \href{https://youtu.be/lwNjuT5e-FM}{\tt Link}

Proposal --$>$ \href{https://youtu.be/2ptUw7MpsMc}{\tt Link} 